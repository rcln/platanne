\frame{

\tableofcontents[sectionstyle=show/hide,hideothersubsections]

}

\subsection{Organisation générale}

\frame{
\frametitle{Vue d'ensemble}
\begin{itemize}
\item Deux modules 
\begin{itemize}
\item \utilsModule : packages non UIMA 
\begin{itemize}
\item {\tt fr.lipn.nlptools.util.*}
\item appel programme externe + alignement / conversions format
\end{itemize}
\item \uimaModule : packages UIMA, dépend de \utilsModule
\begin{itemize}
\item {\tt fr.lipn.nlptools.uima.*}
\item Composants TagEN, TreeTagger, LIA*, YaTeA + ``boîte à outils''
\end{itemize}
\end{itemize}
\end{itemize}

\begin{itemize}
\item 2 ``formes'' pour \uimaModule :
\begin{itemize}
\item Librarie JAR pour déploiement
\item Environnement avec scripts, doc, sources, tests, 
\end{itemize}
\item Conçu pour prise en main progressive
\begin{itemize}
\item de {\em end-user} à développeur UIMA
\end{itemize}
\end{itemize}
}


\subsection{Module indépendant \utilsModule}

\frame{
\frametitle{Appel de programme externe}

\begin{itemize}
\item Transmission ``à la volée''
\begin{itemize}
\item pas de double/triple occupation mémoire
\item temps réduit : lire/écrire simultanément $<$ lire puis écrire
\item pas d'accès disque
\end{itemize}
\item Difficultés
\begin{itemize}
\item Erreurs d'E/S 
%\item Fichier temporaires à éviter si possible
\item Flux d'E/S : risque d'interblocage
\begin{itemize}
\item appelé écrit sur {\tt stdout}, attend lecture par appelant
\item[$\rightarrow$] programme appelant doit lire avant fin du programme !
\end{itemize}
\item[$\Rightarrow$] parallélisme, avec threads Java
\begin{itemize}
\item risques habituels, garantir terminaison dans tous les cas
\item + spécificités UIMA : le CAS n'est pas ``partageable''
\end{itemize}
\end{itemize}
\item Objets {\tt Reader} et {\tt Writer} utilisés (simples, flexibles)
\end{itemize}

}

\frame{
\frametitle{(Ré-)alignement, conversions}
\begin{itemize}
\item Nombreuses conversions CAS $\leftrightarrow$ format programme
\item Modularité : 3 composants génériques
\begin{itemize}
\item {\tt AnnotatedTextReader} lit le texte annoté
\item {\tt InputReader} reçoit chaque token + compare
\item {\tt AlignerConsumer} consomme ces données
\end{itemize}
\item Intérêts : combinaisons de composants, unicité du code
\begin{itemize}
\item paramètres variés pour tous les cas
\item débuggage plus facile
\end{itemize}
\item Objets {\tt Reader} utilisés (simple, flexible)
\item Formats ``un token par ligne'', *ML (balises), positions
\end{itemize}
}


\subsection{Composants UIMA \uimaModule}

\frame{
\frametitle{Boîte à outils}

\begin{itemize}
\item Annotateur générique pour programme encapsulé
\begin{itemize}
\item paramètres communs : langage, chemin, encodage, time out
\item environnement d'exécution, erreurs éventuelles
\end{itemize}
\item Encapsulation d'itérateurs spécifiques :
\begin{itemize}
\item synchronisation (si threads)
\item annotations souvent utilisées ensemble
\begin{itemize}
\item {\tt Token, PartOfSpecch, Lemma} (gestion superposition)
\end{itemize}
\item séries concurrentes : 
\begin{itemize}
\item lecture avec {\em contraintes} sur {\tt componentId}
\item prise en compte des {\tt Interpretation}
\end{itemize}
\end{itemize}
\item Composants généraux :
\begin{itemize}
\item Lecture/écriture du CAS au format ``un token par ligne''
\item Autres utilitaires (prévus !)
\begin{itemize}
\item ex: fusion/décomposition de documents
\end{itemize}
\end{itemize}
\end{itemize}
}


\frame{
\frametitle{Composants encapsulés}

Quelques principes :
\begin{itemize}
\item Conserver au maximum les fonctions du programme
\begin{itemize}
\item paramètres, données en sortie
\end{itemize}
\item Signaler les erreurs au plus tôt
\begin{itemize}
\item éviter d'appeler le programme si paramètre non valide
\item vérifier les caractères spéciaux 
\end{itemize}
\item Utiliser les méthodes officielles
\begin{itemize}
\item Ex: fichier temporaire, questions d'encodage, XML...
\end{itemize}
\item Envisager les différentes utilisations
\begin{itemize}
\item le code doit permettre la parallélisation
\item éviter les suppositions 
\begin{itemize}
\item ex: les documents ne sont pas nécessairement des fichiers
\end{itemize}
\end{itemize}
\end{itemize}
}


