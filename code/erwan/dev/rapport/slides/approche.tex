
\frame{

\tableofcontents[sectionstyle=show/hide,hideothersubsections]

}

\subsection{Objectifs} % plus précis

\frame{
\frametitle{Objectifs}

\begin{itemize}
\item Encapsuler les outils utilisés par l'équipe
\begin{itemize}
\item TagEN (entités nommées), TreeTagger (POS), YaTeA (termes) (+LIA)
\item en faciliter l'utilisation (robustesse, flexibilité)
\begin{itemize}
\item encodages, formats, problèmes d'alignement, etc.
\end{itemize}
\end{itemize}
\item Plateforme {\bf évolutive}
\begin{itemize}
\item proposant des outils de base, dont 
\begin{itemize}
\item composants et applications immédiates
\item ``utilitaires'' de base (pour utilisation actuelle et composants futurs)
\end{itemize}
\item sur laquelle des composants futurs {\em non prévisibles} peuvent s'intégrer
\begin{itemize}
\item éviter restrictions/contraintes $\rightarrow$ généricité
\end{itemize}
\item permettant l'usage d'annotations concurrentes
\end{itemize}
\end{itemize}
}
\frame{
\frametitle{Différents utilisateurs}

\begin{itemize}
\item {\bf Boîte noire.} lancer un CPE (chaîne de traitement) prédéfini, sans contrôle (ou peu) sur les paramètres
\item {\bf Outils non UIMA.} utiliser les librairies indépendantes
\item {\bf UIMA end-user} configurer une chaîne de traitement faite de composants UIMA $\rightarrow$ pré-requis UIMA léger
\item {\bf Programmeur UIMA} construire de nouveaux annotateurs basés sur le même ``c\oe ur'' $\rightarrow$ pré-requis UIMA+\softName 
\item {\bf Maintenance} construire/améliorer la base des composants \softName $\rightarrow$ pré-requis UIMA+\softName sérieux !
\end{itemize}
}

\frame{

\frametitle{Contraintes multiples}

\begin{itemize}
\item Niveau inférieur $\rightarrow$ outils encapsulés
\begin{itemize}
\item pas toujours bien spécifiés
\item pas toujours exempts de bugs
\item limitations intrinsèques
\item prévenir les problèmes liés à l'encapsulation
\end{itemize}
\item Niveau supérieur $\rightarrow$ différents  utilisateurs
\begin{itemize}
\item facilité d'utilisation utilisateur occasionel
\item liberté de paramétrage utilisateur avancé
\item API claire pour programmeur de composants
\item sans/peu contraintes pour composants futurs
\end{itemize}
\item Framework UIMA
\begin{itemize}
\item contraintes techniques/conceptuelles
\end{itemize}
\end{itemize}

}


\subsection{Composants par encapsulation}

\frame{
\frametitle{Appel externe à un outil : difficultés}

\begin{itemize}
\item {\bf Portabilité} abandonnée. 
\begin{itemize}
\item à souligner pour les composants concernés car rare
\end{itemize}
\item {\bf Sûreté} du procecssus UIMA diminuée
\begin{itemize}
\item rupture des mécanismes de contrôle (exceptions, log, ...)
\item failles potentielles du programme externe
\end{itemize}
\item {\bf Facilité d'emploi}  diminuée
\begin{itemize}
\item nécessité d'installer/localiser l'outil externe
\item parfois contraintes inattendues pour cadre UIMA
\item erreurs non filtrées du programme 
\end{itemize}
\item Transmission des {\bf entrées/sorties} 
\begin{itemize}
\item conversions de formats (erreurs, pertes)
\item erreurs ``physiques'' (échec, blocage)
\end{itemize}
\item {\bf Efficacité}
\begin{itemize}
\item perte en temps/espace pour transmission
\item surcharge mémoire (stockage en double)
\end{itemize}
\end{itemize}
}

\subsection{Choix / guidelines}

\frame{

\frametitle{Principe de précaution (bonnes pratiques)}

\begin{itemize}
\item Programmation modulaire
\begin{itemize}
\item Classes/packages dédiées à une tâche (ex: conversions)
\begin{itemize}
\item préférer classes Java officielles
\item ré-utilisabilité du code testé
\item gestion harmonisée des cas/paramètres
\item inconvénient : plus complexe
\item {\em bug-free} impossible $\rightarrow$ faciliter débuggage
\end{itemize}
\end{itemize}
\item Documentation claire et complète
\begin{itemize}
\item spécifications des composants, comportement particulier
\item choix d'implémentation
\end{itemize}
\item Respect des conventions le cas échéant
\begin{itemize}
\item faciliter reprise du code
\item ex: nommage de packages/classes, langue, etc.
\end{itemize}
\end{itemize}
}

\subsection{Type System}

\frame{
\frametitle{Orientation du Type System} % intérêts, expls
\begin{itemize}
\item 2 grandes approches : 
\begin{itemize}
\item Précis, exhaustif
\begin{itemize}
\item typologie riche, attributs prédéfinis
\item complexe mais simple à utiliser 
\item[$\rightarrow$] cadre strict, besoins non prévus non/peu réalisables
\end{itemize}
\item Générique, abstrait.
\begin{itemize}
\item flexible, modulable selon besoin 
\item[$\rightarrow$] plus difficile à {\em bien} utiliser, moins ``propre''
\end{itemize}
\end{itemize}
\item[$\Rightarrow$] TS générique car flexibilité indispensable dans notre cas
\end{itemize}

\begin{itemize}
\item Ensemble restreint de types ``minimaux'', ``peu typés'' :
\begin{itemize}
\item facilite interprétation, {\em mapping} (couples attribut/valeur)
\item extensible localement
\item favorise combinaison d'annotations indépendantes, notamment concurrentes
\end{itemize}
\end{itemize}
}

\frame[plain]{
\frametitle{LIPN Type System}

\noindent\begin{center}
\noindent\hspace*{-.75cm}\scalebox{.8}[.8]{\includegraphics{../LIPN-TS.eps}}
\end{center}

}


\frame{
\frametitle{Annotations concurrentes} % expls

\begin{itemize}
\item Ensemble distincts d'annotations, même texte, rôle similaire
\begin{itemize}
\item ex1: comparaison de différents outils sur la même tâche
\item ex2: on veut annoter par 2 composants A et B;\\ tokenization par A' meilleure pour A, mais par B' pour B %\\ $\rightarrow$ 2 séries de tokens ``parallèles''
\item ex3: ambiguïté locale entre 2 analyses syntaxiques possibles
\end{itemize}
\item Problèmes :
\begin{itemize}
\item Représentation dans le Type System ?
\item Annotateur ``non conscient'' des séries concurrentes ?
%\item[$\rightarrow$] Quand/comment un annotateur a accès à une/des série(s) concurrente(s) ?
\end{itemize}
\item Options envisagées :
\begin{itemize}
\item Feature {\tt componentId} pour distinguer l'annotateur créateur
\item Type {\tt Interpretation} avec feature ``liste d'annotations''
\item Concept des {\em vues} UIMA
\end{itemize}
\end{itemize}

}


